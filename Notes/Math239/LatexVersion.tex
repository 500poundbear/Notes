\documentclass[]{article}
\usepackage[T1]{fontenc}
\usepackage{lmodern}
\usepackage{amssymb,amsmath}
\usepackage{ifxetex,ifluatex}
\usepackage{fixltx2e} % provides \textsubscript
% use upquote if available, for straight quotes in verbatim environments
\IfFileExists{upquote.sty}{\usepackage{upquote}}{}
\ifnum 0\ifxetex 1\fi\ifluatex 1\fi=0 % if pdftex
  \usepackage[utf8]{inputenc}
\else % if luatex or xelatex
  \ifxetex
    \usepackage{mathspec}
    \usepackage{xltxtra,xunicode}
  \else
    \usepackage{fontspec}
  \fi
  \defaultfontfeatures{Mapping=tex-text,Scale=MatchLowercase}
  \newcommand{\euro}{€}
\fi
% use microtype if available
\IfFileExists{microtype.sty}{\usepackage{microtype}}{}
\ifxetex
  \usepackage[setpagesize=false, % page size defined by xetex
              unicode=false, % unicode breaks when used with xetex
              xetex]{hyperref}
\else
  \usepackage[unicode=true]{hyperref}
\fi
\hypersetup{breaklinks=true,
            bookmarks=true,
            pdfauthor={},
            pdftitle={},
            colorlinks=true,
            citecolor=blue,
            urlcolor=blue,
            linkcolor=magenta,
            pdfborder={0 0 0}}
\urlstyle{same}  % don't use monospace font for urls
\setlength{\parindent}{0pt}
\setlength{\parskip}{6pt plus 2pt minus 1pt}
\setlength{\emergencystretch}{3em}  % prevent overfull lines
\setcounter{secnumdepth}{0}

\author{}
\date{}

\begin{document}

\section{Graph Theory Facts and
Propositions}\label{graph-theory-facts-and-propositions}

\subsection{General:}\label{general}

\begin{enumerate}
\def\labelenumi{\arabic{enumi}.}
\item
  Handshake Theorem: $\sum_{v \in V(G)} deg(v) = 2|E|$
\item
  Proposition: Every graph with $\geq 2$ vertices has two vertices of
  the same degree
\item
  Proposition: the n-cube has $2^n$ vertices and $n*2^{n-1}$ edges
\item
  Theorem: if there is a walk from vertex x to vertex y in G then there
  is a path from x to y in G.
\item
  Corollary: if there is a path from x to y in G and a path from y to z
  in G then there is a path from x to z in G.
\item
  Theorem: let G be a graph and let v be a vertex in G. If for each w in
  G there is a path from v to w, then G is connected. \emph{For any
  vertex, you can get to any other vertex.}
\item
  Theorem: a graph G is \textbf{not connected} iff there exists a
  property subset of x of V(G) such that the \textbf{cut} induced by x
  is empty.
\item
  Proposition: if every very has degree $\geq 2$ then G has a cycle.
\item
  Theorem (Dirac): if G is a graph on n \textgreater{} 3 vertices where
  every vertex has degree $\geq \frac{n}{2}$, then G has a cycle
  containing every vertex. G is a \textbf{Hamiltonian Graph}.
\item
  Theorem (Chvtal '72): if G is a graph on n vertices with degree
  $d_1 \leq d_2 \leq d_3 ... \leq d_n$ then if $d_i \geq i$ or
  $d_{n-i} \geq n - i$ for all $i \leq \frac{n}{2}$, then G is
  Hamiltonian.
\item
  Theorem (Tutte): Every 4-connected graph that can be drawn in the
  plane without crossings is Hamiltonian.
\item
  Theorem: Every connected graph in which every vertex has even degree
  is \emph{Eulerian}. An Eulerian graph has an Euler tour, which is a
  closed walk that contains every edge once.
\item
  Lemma: if e = \{x,y\} is a bridge of a connected graph G, then G-e has
  precisely two components. Furthermore, x and y are in different
  components.
\item
  Theorem: An edge is a bridge of a graph G iff it is not contained in a
  cycle of G
\item
  Corollary: If there are two distinct paths from u to v in G then G
  contains a cycle.
\item
  Lemma: There is a unique path between every pair of vertices u and v
  in a tree.
\item
  Lemma: Every edge of a tree T is a bridge.
\item
  Theorem: A tree with at least 2 vertices has at least two vertices of
  degree 1.
\item
  Theorem: if T is a tree, then \textbar{}E(T)\textbar{} =
  \textbar{}V(T) - 1\textbar{}.
\item
  Proposition: If x,y are vertices of a tree T, then there is a unique
  path of T from x to y.
\item
  Theorem: if T is a tree, then \textbar{}E(T)\textbar{} =
  \textbar{}V(T) - 1\textbar{}.
\item
  Proposition: Every edge of a tree is a bridge.
\item
  Proposition: If x,y are vertices of a tree T, then there is a unique
  path of T from x to y.
\item
  Proposition: A graph G has a spanning tree iff it is connected.
\item
  Corollary: Every connected graph on n vertices has $\geq n-1$ edges.
\item
  Corollary: Every connected graph on n vertices, n-1 edges is a tree.
\item
  Proposition: Every tree is bipartite.
\item
  Proposition: If G is a bipartite graph and u,v $\in V(G)$ then if u
  and v are in the same part of a bipartition, then every walk from u to
  v has even length. If u,v are in different parts, then every walk from
  u to v has odd length.
\item
  Proposition: If G is a graph with no odd cycles, then G is bipartite.
\item
  Theorem: Prim's algorithm outputs a min-weight spanning tree.
\item
  Proposition: A graph is planar iff it has a spherical embedding.
\item
  Theorem: if there is a planar embedding of 2-connected graph G with
  faces $f_1, f_2, ...$ then $\sum_{i=1} deg(f_i) = 2|E(G)|$
\item
  Corollary: If the connected graph G has a planar embedding with f
  faces, then average degree of a face is $\frac{2|E(G)|}{f}$.
\item
  Theorem: let G be a connected graph with \textbar{}V\textbar{}
  vertices and \textbar{}E\textbar{} edges. If G has a planar embedding
  with \textbar{}F\textbar{} faces, then \textbar{}V\textbar{} -
  \textbar{}E\textbar{} + \textbar{}F\textbar{} = 2.
\item
  Theorem: There are exactly five non-isomorphic platonic solids.
\item
  Lemma: Let G be a planar embedding with \textbar{}V\textbar{}
  vertices, \textbar{}E\textbar{} edges and \textbar{}F\textbar{} faces.
  Then \{d,k\} is one of the five pairs of faces and vertices: \{3,3\},
  \{3,4\}, \{4,3\}, \{5,3\}, \{3,5\}
\item
  Lemma: If G is connected and not a tree then in a planar embedding of
  G, the boundary of each face contains a cycle.
\item
  Lemma: Let G be a planar embedding with \textbar{}V\textbar{} vertices
  and \textbar{}E\textbar{} edges. If each face has degree at least d,
  then (d-2)\textbar{}E\textbar{} $\leq$ d(\textbar{}V\textbar{}-2)\$.
\item
  Corollary: In any planar embedding of a graph with at least 2 faces,
  each face has degree $\geq 3$.
\item
  Lemma: In any planar embedding of a graph with $\geq$ 1 cycle, the
  boundary of every face contains a cycle.
\item
  Lemma (Test 1): If G = (V,E) is a planar graph and
  \textbar{}E\textbar{} $\geq 2$, then \textbar{}E\textbar{} $\leq$
  3\textbar{}V\textbar{}-6.
\item
  Corollary: $K_5$ is non-planar \textbar{}V\textbar{} = 5,
  \textbar{}E\textbar{} = 10.
\item
  Corollary: A planar graph has a vertex of degree at most 5.
\item
  Lemma (Test 2): If G = (V,E) is a planar graph and every cycle has
  length $\geq$ g, where g is the girth, the length of the smallest
  cycle, and \textbar{}E\textbar{} $\geq \frac{1}{2}g$, then
  $|E| \leq \frac{g}{g-2}(|V| - 2)$
\item
  Corollary: $K_{3,3}$ is non-planar because it has no triangles, so g =
  4 and it fails Test 2.
\item
  Kuratowski's Theorem: A graph is planar iff it has no subdivision of
  $K_{3,3}$ or $K_5$ as a subgraph.
\item
  Theorem: A graph is 2-colourable iff it is bipartite.
\item
  Theorem: $K_n$ is n-colourable and not k-colourable for k \textless{}
  n.
\item
  Five-Colour-Theorem: Every planar graph is 5-colourable.
\item
  Theorem: Every planar graph is 4-colourable.
\item
  Lemma: M is not a maximum matching iff there exists an M-augmenting
  path.
\item
  Lemma: If M is a matching of G and C is a cover of G then
  $|M| \leq |C|$.
\item
  Lemma: If M is matching and C is a cover and \textbar{}M\textbar{} =
  \textbar{}C\textbar{} then M is a maximum matching and C is a minimum
  cover.
\item
  Theorem (Konig's Theorem): If G is bipartite, then the size of the
  maximum matching is equal to the size of the minimum cover.
\item
  Lemma: Let G be a bipartite graph with bipartition A,B where
  \textbar{}A\textbar{} = \textbar{}B\textbar{} = n. If G has
  \textbar{}E\textbar{} edges then G has a matching of at least size
  $\frac{q}{n}$.
\item
  Theorem (Hall's): An (A,B)-bigraph G has a matching that saturates A
  iff for every S subset of A, \textbar{}S\textbar{} $\leq$
  \textbar{}N(S)\textbar{}.
\item
  Corollary: An (A,B) bigraph G has a perfect matching iff
  \textbar{}A\textbar{}=\textbar{}B\textbar{} and for S is a subset of
  A, \textbar{}S\textbar{} $\leq$ \textbar{}N(S)\textbar{}.
\item
  Proposition: If k $\geq 1$ and G is a k-regular bipartite graph, then
  G has a perfect matching.
\item
  Corollary: If G is a k-regular bipartite graph, then E(G) has a
  partition into k perfect matches of G.
\item
  Corollary: Following from right above, every k-regular bipartite graph
  is k-edge colourable.
\end{enumerate}

\end{document}
